\chapter{Group}

\section{Basic}

\begin{definition}[group]
\end{definition}

\begin{proposition}[2nd definition]
\end{proposition}

\begin{proposition}[3rd definition]
\end{proposition}

\begin{definition}[order of group]
\end{definition}

\begin{definition}[order of an element in group]
\end{definition}

\begin{definition}[permutation group]
\end{definition}

\begin{notation}
$S_n$ permutation group
\end{notation}

\begin{definition}[cyclic group]
\end{definition}

\section{Subgroup \& Coset}

\begin{definition}[subgroup]
\end{definition}

\begin{notation}
$H \leq G$ subgroup
\end{notation}

\begin{proposition}[intersection of subgroups is still a subgroup]
\end{proposition}

\begin{definition}[center]
\end{definition}

\begin{definition}[subgroup generator]
\end{definition}

\begin{notation}[representation of subgroup generated by a subset]
\end{notation}

\begin{theorem}[representation of subgroup generated by a subset]
\end{theorem}

\begin{theorem}[every cyclic group can be generated by a singleton subset]
\end{theorem}

\begin{warning}
Infinite group can be generated by a finite subset. 
\end{warning}

\begin{example}
$(\mathbb{Z}, +) = \langle\{1\}\rangle$
\end{example}

\begin{theorem}[order of the group which is generated by a singleton subset is the same as the order of the element]
\end{theorem}





