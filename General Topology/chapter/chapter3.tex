\chapter{Metric Space}

\section{Metric Space}

\begin{definition}
(metric)
\end{definition}

\begin{definition}
(metric space)
\end{definition}

\begin{definition}
(cauchy sequence)
\end{definition}

\begin{definition}
(convergence of sequence)
\end{definition}

\begin{definition}
(complete space)
\end{definition}

\begin{definition}
(bounded space)
\end{definition}

\begin{definition}
(totally bounded space)
\end{definition}

\begin{warning}
Not every bounded space is a totally bounded space. 
\end{warning}

\begin{theorem}
Every metric space $(X, d)$ can generate a topology space $(X, \mathcal{T}_{d})$. 
\end{theorem}

\begin{theorem}
Given a compact metric space $(X, d)$ and $Y \subseteq X$. If $Y$ is closed, then $Y$ is compact. 
\end{theorem}

\begin{theorem}
Given metric space $(X, d)$, the following statements are equivalent: 
\end{theorem}

\begin{theorem}
Given a metric space $(X, d)$ and $Y \subseteq X$. $Y$ is a compact space if and only if $Y$ is complete and totally bounded. 
\end{theorem}

\begin{definition}
(dense set)
\end{definition}

\begin{definition}
(separable set)
\end{definition}

\section{Normed Vector Space}

\begin{definition}
(norm)
\end{definition}

\begin{definition}
(normed vector space)
\end{definition}

\begin{definition}
(Banach space)
\end{definition}

\begin{theorem}
Every normed vector space $(X, \Vert\cdot\Vert)$ can generate a metric space $(X, \Vert\cdot\Vert_{d})$. 
\end{theorem}

\section{Map and Function}

\begin{definition}
(pointwise continuity)
\end{definition}

\begin{definition}
(uniformly continuity)
\end{definition}

\begin{theorem}
Uniformly continuity implies pointwise continuity. 
\end{theorem}

\begin{theorem}
(Dini's theorem)
\end{theorem}







