\chapter{Topology Space}

%\cite{THM}

\begin{definition}
(open set)
\end{definition}

\begin{definition}
(closed set) Complement of open set. 
\end{definition}

\begin{definition}
(topology space)
\end{definition}

\begin{definition}
(subspace)
\end{definition}

\begin{definition}
(basis)
\end{definition}

\begin{definition}
(neighborhood)
\end{definition}

\begin{definition}
(limit point)
\end{definition}

\begin{definition}
(derived set)
\end{definition}

\begin{definition}
(adherent point)
\end{definition}

\begin{definition}
(isolation point)
\end{definition}

\begin{definition}
(interior point)
\end{definition}

\begin{definition}
(boundary point)
\end{definition}

\begin{warning}
limit point $\implies$ adherent point, but adherent point $\centernot\implies$ limit point.
\end{warning}

\begin{example}
to-do
\end{example}

\begin{theorem}
Given topology space $(X, \mathcal{T})$ and $Y \subseteq X$. Y is closed if and only if Y is derived set. 
\end{theorem}

\begin{proof}
Proof by contradiction. 
\end{proof}

\begin{definition}
(closure)
\end{definition}

\begin{theorem}
Given topology space $(X, \mathcal{T})$, the following statements are equivalent: 
\end{theorem}

\begin{definition}
(open cover)
\end{definition}

\begin{definition}
(compact space)
\end{definition}

\begin{definition}
(precompact space)
\end{definition}


















