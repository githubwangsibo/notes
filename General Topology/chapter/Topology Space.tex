\chapter{Topology Space}

\section{Topology Space}

\begin{definition}
(open set)
\end{definition}

\begin{definition}
(closed set) Complement of open set. 
\end{definition}

\begin{definition}
(topology space)
\end{definition}

\begin{definition}[topological subspace]
\end{definition}

\begin{theorem}[subspace generated by a subset]
\end{theorem}

\begin{definition}
(basis)
\end{definition}

\begin{definition}
(neighborhood)
\end{definition}

\section{Set}

\begin{definition}
\label{definition:limit point of set}
(limit point of set)
\end{definition}

\begin{definition}
(derived set)
\end{definition}

\begin{definition}
(adherent point)
\end{definition}

\begin{definition}
(isolation point)
\end{definition}

\begin{definition}
(interior point)
\end{definition}

\begin{definition}
(boundary point)
\end{definition}

\begin{warning}
limit point $\implies$ adherent point, but adherent point $\centernot\implies$ limit point.
\end{warning}

\begin{example}
to-do
\end{example}

\begin{theorem}
Given topology space $(X, \mathcal{T})$ and $Y \subseteq X$. Y is closed if and only if Y is derived set. 
\end{theorem}

\begin{proof}
Proof by contradiction. 
\end{proof}

\begin{definition}
(closure)
\end{definition}

\begin{theorem}
Given topology space $(X, \mathcal{T})$, the following statements are equivalent: 
\end{theorem}

\begin{definition}[dense set]
\end{definition}

\begin{definition}[separable space]
\end{definition}

\section{Open Cover}

\begin{definition}
(open cover)
\end{definition}

\begin{definition}
(compact space)
\end{definition}

\begin{definition}
(precompact space)
\end{definition}

\begin{theorem}
Given a topology space $(X, \mathcal{T})$ and $Y, K \subset X$. If $Y$ is closed and $K$ is compact, then $Y \cap K$ is compact. 
\end{theorem}

\begin{theorem}[Cantor’s intersection theorem]
\end{theorem}

\begin{proof}
to do
\end{proof}

\section{Sequence}

\begin{definition}
(sequence)
\end{definition}

\begin{definition}
(subsequence)
\end{definition}

\begin{definition}
\label{definition:limit point of sequence}
(limit point of sequence)
\end{definition}

\begin{warning}
may not be the same. 
\end{warning}

\begin{example}
to do
\end{example}

\begin{definition}
\label{definition:limit of sequence}
(limit of sequence)
\end{definition}

\begin{warning}
may not be the same. 
\end{warning}

\begin{example}
to do
\end{example}

\begin{warning}
In some topology space, the limit of a sequence may not be unique. 
\end{warning}

\begin{example}
to do
\end{example}

\begin{theorem}
In any metric space, the limit of any sequence is unique. 
\end{theorem}

\begin{proof}
Proof by contradiction. 
\end{proof}

\begin{theorem}
Given a topology space $(X, \mathcal{T})$ and the sequence $\{x_n\} \in X$. If $\{x_n\}$ converges to $x \in X$, then any subsequence $\{y_n\}$ in $\{x_n\}$ converges to $x$ as well. 
\end{theorem}

\begin{corollary}
Given a topology space $(X, \mathcal{T})$ and the sequence $\{x_n\} \in X$. Let $\{y_n\}$ and $\{z_n\}$ denote two different subsequence in $\{x_n\}$. If $\{y_n\}$ and $\{z_n\}$ converge to $y \in X$ and $z \in X$ respectively with $y \neq z$, then $\{x_n\}$ is not a convergent sequence. 
\end{corollary}

\section{Metric Space}

\begin{definition}
(metric)
\end{definition}

\begin{definition}
(metric space)
\end{definition}

\begin{theorem}
Every metric space $(X, d)$ can generate a topology space $(X, \mathcal{T}_{d})$. 
\end{theorem}

\begin{theorem}
Any compact metric space is separable. 
\end{theorem}

\begin{theorem}
Given a compact metric space $(X, d)$ and $Y \subseteq X$. If $Y$ is closed, then $Y$ is compact. 
\end{theorem}

\begin{definition}
(Cauchy sequence)
\end{definition}

\begin{definition}
(convergence of sequence)
\end{definition}

\begin{theorem}
Given a Cauchy sequence. If it has a convergent subsequence, then the Cauchy sequence is convergent. 
\end{theorem}

\begin{proof}
\end{proof}

\begin{definition}
(complete space)
\end{definition}

\begin{theorem}
Given a complete metric space $(X, d)$ and a subset $Y \subset X$, $Y$ is complete if and only if $Y$ is closed. 
\end{theorem}

\begin{warning}
Given a non-complete metric space $(X, d)$ and a subset $Y \subset X$, $Y$ is complete implies $Y$ is closed, but $Y$ is closed cannot imply $Y$ is complete. 
\end{warning}

\begin{example}
to do
\end{example}

\begin{definition}[sequential compact]
\end{definition}

\begin{definition}
(bounded space)
\end{definition}

\begin{definition}
(totally bounded space)
\end{definition}

\begin{warning}
Not every bounded space is a totally bounded space. 
\end{warning}

\begin{example}
to do
\end{example}

\begin{theorem}
totally bounded $\implies$ separable. 
\end{theorem}

\begin{theorem}
Given a metric space $(X, d)$, the following statements are equivalent: 
\end{theorem}

\begin{lemma}
sequential compact $\implies$ totally bounded. 
\end{lemma}

\begin{proof}
We prove it by contradiction. Suppose it is not totally bounded, then $\exists \varepsilon$ such that $X$ cannot be covered by finite open balls.

So we can find an infinite sequence $\{x_i\}$ in $X$ such that $d(x_i, x_j) \geq \varepsilon$, $\forall i, j \in \mathbb{N}$ and $i \neq j$. Otherwise, $X$ is totally bounded.

Hence $\{x_i\}$ has no convergent subsequence. So $X$ is not sequential compact, which is a contradiction. 
\end{proof}

\begin{lemma}
sequential compact $\implies$ complete. 
\end{lemma}

\begin{proof}
\end{proof}

\begin{lemma}
Given a metric space $(X, d)$ and $Y \subseteq X$. $Y$ is compact $\iff$ $Y$ is sequential compact. 
\end{lemma}

\begin{proof}
\end{proof}

\begin{lemma}
Given a metric space $(X, d)$ and $Y \subseteq X$. $Y$ is compact $\iff$ $Y$ is complete and totally bounded. 
\end{lemma}

\begin{proof}
\end{proof}

\begin{theorem}
Given a metric space $(X, d)$ and $Y \subseteq X$. $Y$ compact space $\iff$ $Y$ is sequential compact $\iff$ $Y$ is complete and totally bounded. 
\end{theorem}

\begin{proof}
\end{proof}

\section{Normed Vector Space}

\begin{definition}[norm]
\end{definition}

\begin{definition}[normed vector space]
\end{definition}

\begin{theorem}
Every normed vector space $(X, \Vert\cdot\Vert)$ can generate a metric space $(X, \Vert\cdot\Vert_{d})$. 
\end{theorem} 

\begin{definition}[linear subspace, vector subspace]
\end{definition}

\begin{corollary}
Given a normed vector space $(X, \Vert\cdot\Vert)$. Every singleton set in $X$ is closed. 
\end{corollary}

\begin{definition}[equivalent norm]
\end{definition}

\begin{corollary}[equivalent norm is an equivalent relation]
\end{corollary}

\begin{lemma}[$\Vert \cdot \Vert_{1}$ is a norm]
\end{lemma}

\begin{notation}[$L_1$ norm]
$\Vert \cdot \Vert_{1}$ 
\end{notation}

\begin{lemma}\label{lemma:Any norm is continuous under L1}
In any finite normed vector space, any norm $\Vert \cdot \Vert_{a}$ is continuous under $\Vert \cdot \Vert_{1}$. 
\end{lemma}

\begin{proof}
It suffice to prove that $\forall \varepsilon \in \mathbb{R}^+$, $\exists \delta(\varepsilon) \in \mathbb{R}^+$, such that $\left\Vert x_1 - x_2 \right\Vert_{1} < \delta \implies \left\vert \left\Vert x_1 \right\Vert_{a} - \left\Vert x_2 \right\Vert_{a} \right\vert < \varepsilon$, $\forall x_1, x_2 \in X$. 

Firstly, it is obvious that $\forall x_1, x_2 \in X$, $\left\vert \left\Vert x_1 \right\Vert_{a} - \left\Vert x_2 \right\Vert_{a} \right\vert < \left\Vert x_1 - x_2 \right\Vert_{a}$. 

Then $\forall x_1, x_2 \in X$: 
\begin{align*}
\left\Vert x_1 - x_2 \right\Vert_{a}
&\leq \sum_{i=1}^{n} \left\vert \alpha_1^{i} - \alpha_2^{i} \right\vert \left\Vert e^{i} \right\Vert_{a} \\
&\leq \max_i \{\left\Vert e^{i} \right\Vert_{a} \} \sum_{i=1}^{n} \left\vert \alpha_1^{i} - \alpha_2^{i} \right\vert \\
&\leq \left\Vert x_1 - x_2 \right\Vert_{1} \max_i \{ \left\Vert e^{i} \right\Vert_{a} \}
\end{align*}

Define $\delta \coloneqq \frac{\varepsilon}{\max_i \{ \left\Vert e^{i} \right\Vert_{a} \} }$. It is well defined, due to finite normed vector space. 

Therefore, $\forall \varepsilon \in \mathbb{R}^+$, $\exists \delta \in \mathbb{R}^+$, such that $\left\Vert x_1 - x_2 \right\Vert_{1} < \delta \implies \left\vert \left\Vert x_1 \right\Vert_{a} - \left\Vert x_2 \right\Vert_{a} \right\vert < \varepsilon$, $\forall x_1, x_2 \in X$. 
\end{proof}

\begin{theorem}
In every finite normed vector space, all norms are equivalent. 
\end{theorem}

\begin{proof}
Define $S \coloneqq \{x \in X \colon \Vert x \Vert_1 = 1\}$. Because $S$ is closed and bounded in finite normed vector space $X$, $S$ is compact. 

It suffices to prove that $\exists C_{min}, C_{max} \in \mathbb{R}^+$ such that $\forall x \in S$, $C_{min} \leq \Vert x \Vert_{a} \leq C_{max}$. 

Due to \hyperref[lemma:Any norm is continuous under L1]{Lemma \ref*{lemma:Any norm is continuous under L1}}, $\Vert \cdot \Vert_a$ is a continuous function. Because of Extreme Value Theorem, it has minimum $A_{min}$ and maximum $A_{max}$ on compact set $S$. In another word, $A_{min} \leq \Vert \cdot \Vert_{a} \leq A_{max}$.
\end{proof}

\begin{definition}
(Banach space)
\end{definition}

\section{Inner Product Space}

\begin{definition}[inner product space]
\end{definition}

\begin{definition}[Hilbert space]
\end{definition}







