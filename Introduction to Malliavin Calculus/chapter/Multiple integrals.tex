\chapter{Multiple integrals}

\begin{definition}[elementary functions over $T^n$]
    
\end{definition}

\begin{notation}
    Let $\mathcal{E}_n$ denote the set of all  elementary functions over $T^n$. 
\end{notation}

\begin{definition}[symmetric elementary functions]
    
\end{definition}

\begin{definition}[integral of elementary functions]
    
\end{definition}

\begin{proposition}
    If $f \in \mathcal{E}_n$, then $I_n(f) = I_n(\tilde{f})$. 
\end{proposition}

\begin{proof}
    
\end{proof}

\begin{proposition}
    If $f \in \mathcal{E}_n$ and $g \in \mathcal{E}_m$
    , then $\E[I_n(f) I_m(g)] = \langle \tilde{f}, \tilde{g} \rangle_{L^2}$, 
    when $m=n$, and $=0$ otherwise. 
\end{proposition}

\begin{proof}
    
\end{proof}

\begin{theorem}
    $\mathcal{E}_n$ is dense in $L^2(T^n)$. 
\end{theorem}

\begin{proof}
    
\end{proof}

\begin{proposition}
    \label{proposition:cauchy sequence in L^2}
    Given $f \in L^2(T^n)$ and $\{f_k\}_{k} \in \mathcal{E}_n$, 
    such that $f_k \xrightarrow[]{L^2(T^n)} f$. 
    Then $\{I_n(f_k)\}_k$ is a Cauchy sequence in $L^2(\Omega)$.  
\end{proposition}

\begin{proof}
    
\end{proof}

\begin{definition}[n-th multiple integrals]
    
\end{definition}

\begin{remark}
Because of 
\hyperref[proposition:cauchy sequence in L^2]{Proposition \ref*{proposition:cauchy sequence in L^2}} 
definition is well defined. 
\end{remark}

\begin{theorem}
    If $f \in L^2(T^n)$ and $g \in L^2(T^m)$
    , then $\E[I_n(f) I_m(g)] = \langle \tilde{f}, \tilde{g} \rangle_{L^2}$, 
    when $m=n$, and $=0$ otherwise. 
\end{theorem}

\begin{definition}[iterated Ito integral]
    $\text{Ito}_n(f)$
\end{definition}

\begin{theorem}
    $I_n(f) = \text{Ito}_n(f)$. 
\end{theorem}

\begin{proof}
    
\end{proof}

\begin{theorem}
    Hermite polynomial. 
\end{theorem}

\begin{proof}
    
\end{proof}

\begin{theorem}
    Surjective
\end{theorem}

\begin{proof}
    
\end{proof}

\begin{theorem}
    decomposition. 
\end{theorem}

\begin{proof}
    
\end{proof}
