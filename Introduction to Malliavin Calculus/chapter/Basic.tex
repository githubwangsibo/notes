\chapter{Basic}

The following content can be found in 
\cite{TheMalliavinCalculusandRelatedTopics} and 
\cite{IntroductiontoMalliavinCalculus}. 
For simplicity, we only consider 1-dimensional case. 

\begin{definition}[centered Gaussian family]
Given a complete probability space $(\Omega, \mathcal{F}, \Prob)$, 
a subspace $W \subset$ $L^2(\Omega, \mathcal{F}, \Prob)$ is a centered Gaussian family, if it is closed, and all the elements of $W$ are
Gaussian random variables with zero mean. 
\end{definition}

%\begin{lemma}
%$L^2(\Omega, \mathcal{F}, P)$ is a complete normed vector space. 
%\end{lemma}

\begin{definition}[isonormal Gaussian process]
A centered Gaussian family $W$ on $H$ is called isonormal Gaussian process, 
if $H$ is a real and separable Hilbert space, 
and $W = \{W(h): h \in H \}$
, and $\E[W(f)W(g)] = \langle f, g \rangle_{H}$, $\forall f, g \in H$. 
\end{definition}

\begin{example}
Given a Brownian motion $B_t$ with respect to $(\Omega, \mathcal{F}, \Prob)$. 
$H = L^2([0, T], \mathcal{B}(0, T))$, 
$W: h \mapsto \int_{0}^{T} h(s) d B_s$. 
\end{example}

\begin{property}
    The map $h \mapsto W(h)$ is linear. 
\end{property}

\begin{proof}
    One can verify that $\forall f, g \in H$, $\forall a, b \in \mathbb{R}$
    , $\E[(W(af+bg) - aW(f) - bW(f))^2] = 0$ . 
\end{proof}
