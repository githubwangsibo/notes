\chapter{Derivative operator}

\begin{notation} Define 
$\gls*{mathcalS}$
\[\mathcal{S} \coloneqq 
\{f(W(h_1), \dots, W(h_n)) : \forall n \in \mathbb{N}, f \in C_{pol}^{\infty}(\mathbb{R}^n, \mathbb{R}), h_1, \dots, h_n \in H
\}\]
\end{notation}

\begin{example}
$e^x \notin C_{pol}^{\infty}(\mathbb{R}, \mathbb{R})$
and $x^3 \in C_{pol}^{\infty}(\mathbb{R}, \mathbb{R})$. 
\end{example}

\begin{notation} Define $\mathcal{S}_{0}$
\[\mathcal{S}_{0} \coloneqq 
\{f(W(h_1), \dots, W(h_n)) : \forall n \in \mathbb{N}, f \in C_{0}^{\infty}(\mathbb{R}^n, \mathbb{R}), h_1, \dots, h_n \in H
\}\]
\end{notation}

\begin{example}
$x^3 \notin C_{0}^{\infty}(\mathbb{R}, \mathbb{R})$
and $1 \notin C_{0}^{\infty}(\mathbb{R}, \mathbb{R})$. 
\end{example}

\begin{proposition}
\label{proposition:smoothing varphi}
$C_{0}^{\infty}(\mathbb{R}^n, \mathbb{R}) \neq \emptyset$. 
\end{proposition}

\begin{proof}
$\varphi(x) \coloneqq e^{- \frac{1}{1- \lvert x \rvert^2}} $, 
if $\lvert x \rvert < 1$, 
and $\coloneqq 0$, otherwise. 
\end{proof}

\begin{lemma}
$\mathcal{S}_{0}$ is dense in $\mathcal{S}$, with $L^2(\mathcal{G})$, where $\mathcal{G} \coloneqq \sigma(S)$. 
\end{lemma}

\begin{proof}
Using the similar technique as Lemma 4.3.1 (p. 50) in 
\parencite{S}. 

Because $H$ is a separable Hilbert space, 
$\exists \{h_n\}_{n \in \mathbb{N}} \in H$, such that $\{h_n\}_{n}$ is dense in  $H$. Define $A_n \coloneqq \sigma(W(h_i): i \leq n)$, then $A_n \subset A_{n+1} \subset \mathcal{G}$ and $\lim_{n \to +\infty} A_n = \mathcal{G}$. 

So $\forall F \in \mathcal{S}$ with $L^2(\mathcal{G})$, 
we have $\E[F | \mathcal{A}_n] 
\xrightarrow[n \to +\infty]{L^2} 
\E[F | \mathcal{G}] = F $, 
from Corollary C.9 (p. 325) in 
\parencite{S}. 

From \hyperref[theorem:Doob-Dynkin Lemma]
{Theorem \ref*{theorem:Doob-Dynkin Lemma}}, 
$\forall n \in \mathbb{N}$, $\exists F_n \in L^2(\mathbb{R}^n, \mathbb{R})$, 
such that 
\[\E[F | \mathcal{A}_n] = F_n(W(h_1), \dots, W(h_n))\]

In addition, we know that $\forall n \in \mathbb{N}, C_{0}^{\infty}(\mathbb{R}^n, \mathbb{R})$ is dense in $L^2(\mathbb{R}^n, \mathbb{R})$, 
from \hyperref[theorem:compact support smooth function is dense in L^2]
{Theorem \ref*{theorem:compact support smooth function is dense in L^2}}. 

Finally, $F$ can be approximated by some elements of $\mathcal{S}_{0}$ in $L^2$, by Minkowski's inequality.  
\end{proof}

\begin{definition}[derivative]
Derivative operator $D : \mathcal{S} \to L^2(\Omega, H, \Prob)$, 
$DF = \sum_{i=1}^{n} \frac{\partial }{\partial x_i} f(W(h_1), \dots, W(h_n)) h_i$. 
\end{definition}

\begin{lemma}
\label{lemma:rule}
$\forall F, G \in \mathcal{S}$, 
$D(FG) = F(DG) + G(DF)$. 
\end{lemma}

\begin{proof}
Multiplication of smooth functions is still a smooth function. Then using the definition. 
\end{proof}

\begin{proposition}
\label{proposition:one variable}
$\forall F \in \mathcal{S}$, $\forall h \in H$, 
$\E[\langle DF, h \rangle_{H}] 
= \E[FW(h)]$. 
\end{proposition}

\begin{proof}
Without loss of generality, we only prove the case 
when 
\[F = f(W(h), \dots, W(e_n))\] 
and $\lVert h \rVert_{H} = 1$, 
where $h, e_1, \dots, e_n$ are orthogonal. 

Thus, 
\begin{equation*}
\begin{aligned}
&\E[\langle DF, h \rangle_{H}]  \\
= &\E[\frac{\partial}{\partial x_1} f(W(h), \dots, W(e_n))]
&\text{} \\
= &\int_{\mathbb{R}^n} \frac{\partial}{\partial x_1} f(x_1, \dots, x_n) \phi(x_1, \dots, x_n) d x_1\dots d x_n 
&\text{} \\
= &\int_{\mathbb{R}^n} f(x_1, \dots, x_n) \phi(x_1, \dots, x_n) x_1 d x_1\dots d x_n 
&\text{} \\
= &\E[f(W(h), \dots, W(e_n))W(h)]
&\text{} \\
= &\E[FW(h)]
&\text{} \\
\end{aligned}
\end{equation*}

where $\phi$ is the density of $n$ variable standard normal distribution.
\end{proof}

\begin{proposition}
\label{proposition:2 terms}
$\forall F, G \in \mathcal{S}$, $\forall h \in H$, 
\[\E[G \langle DF, h \rangle_{H}] 
+  \E[F \langle DG, h \rangle_{H}] 
= \E[FGW(h)]\]
\end{proposition}

\begin{proof}
Define $X \coloneqq FG$, then $X \in \mathcal{S}$. 
Using \hyperref[lemma:rule]
{Lemma \ref*{lemma:rule}}
and 
\hyperref[proposition:one variable]
{Proposition \ref*{proposition:one variable}}. 
\end{proof}

\begin{notation}
$\forall f \in \mathcal{S}$, 
$\lVert f \rVert_{1, 2}^2 \coloneqq 
\lVert f \rVert_{L^2(\Omega)}^2
+ \lVert Df \rVert_{L^2(\Omega, H)}^2$
\end{notation}

\begin{proposition}
\label{proposition:Soblev norm}
$\lVert \cdot \rVert_{1, 2}$ is a norm. 
\end{proposition}

\begin{proof}
To prove sub-additivity, we will apply Minkowski's inequality. 
\end{proof}

\begin{notation}
Let $(\mathbb{D}^{1, 2}, \lVert \cdot \rVert_{1, 2})$ 
be the closure of $(\mathcal{S}, \lVert \cdot \rVert_{1, 2})$. 
\end{notation}

\begin{lemma}
\label{lemma:D is well defined}
Given $\{X_n\}_n \in \mathcal{S}$, 
if $X_n \xrightarrow{L^2(\Omega)} 0$, 
and $DX_n \xrightarrow{L^2(\Omega, H)} U$, 
then $U = 0$ almost surely.
\end{lemma}

\begin{proof}
We know that 
\[\lim_{n \to +\infty} \E[X_n^2] = 0\]
and 
\[\lim_{n \to +\infty} 
\E[ \langle DX_n - U , DX_n - U \rangle_{H} ] = 0\]

$\forall h \in H$ and $\forall F \in \mathcal{S}_{0}$,  
\begin{equation*}
\begin{aligned}
&\E[F \langle U, h \rangle_{H}]  \\
= &\lim_{n \to +\infty} \E[F \langle DX_n, h \rangle_{H}]
&\text{(continuity of inner product)} \\
= &\lim_{n \to +\infty} 
\E[- X_n \langle DF, h \rangle_{H}
+ X_n F W(h)]
&\text{(\hyperref[proposition:2 terms]
{Proposition \ref*{proposition:2 terms}})} \\
= & 0
&\text{(Cauchy–Schwarz inequality)} \\
\end{aligned}
\end{equation*}

Because $F \in \mathcal{S}_{0}$, we know $\langle U, h \rangle_{H} = 0$, 
from \hyperref[theorem:zero is the only vector orthogonal to dense]
{Theorem \ref*{theorem:zero is the only vector orthogonal to dense}}. 
Also, because $h \in H$, $U = 0$ almost surely. 
\end{proof}

\begin{theorem}
Derivative operator $D: \mathbb{D}^{1, 2} \to L^2(\Omega, H, \Prob)$ is well defined.  
\end{theorem}

\begin{proof}
From \refnum{Lemma}{lemma:D is well defined},
we know that $\forall X \in \mathbb{D}^{1, 2}$, $\exists \{X_n\} \in \mathcal{S}$
, \\
such that $X_n \xrightarrow{L^2(\Omega)} X$,
$DX_n \xrightarrow{L^2(\Omega, H)} U$
and $U \in L^2(\Omega, H)$, then $DX \coloneqq \lim_{n \to +\infty} DX_n$. 
\end{proof}

\begin{theorem}[Chain rule]
Given $g \in C^{1}(\mathbb{R}^d, \mathbb{R})$ 
with bounded partial derivatives
and $F_i \in \mathbb{D}^{1, 2}$, $i \in \{1, \dots, d\}$. 
Then $g(F_1, \dots, F_d) \in \mathbb{D}^{1, 2}$ 
and 
\[D(g(F_1, \dots, F_d))
= \sum_{i=1}^{d} \frac{\partial}{\partial x_i}g(F_1, \dots, F_d) DF_i\]
\end{theorem}
\setcounter{stepCounter}{1}
\begin{proof}
For simplicity, we only prove the case when $d= 1$. 

\step{} When $F \in \mathcal{S}$. 

Because the composition of differentiable function and smooth function is still a smooth function, 
we have $g(F) \in \mathcal{S} \subset \mathbb{D}^{1, 2}$ and the chain rule can be obtained easily. 

\step{} When $F \notin \mathcal{S}$. 

$\exists \{F_k\}_{k \in \mathbb{N}} \in \mathcal{S}$, 
such that $F_k \xrightarrow{L^2(\Omega)} F$ 
and $DF_k \xrightarrow{L^2(\Omega, H)} DF$. 

In addition, $\forall \varepsilon > 0$, 
define $\varphi_{\varepsilon} \coloneqq \frac{1}{\varepsilon}\varphi(\frac{x}{\varepsilon})$, 
where $\varphi(x)$ is the same as 
%\hyperref[proposition:smoothing varphi]
%{Lemma \ref*{proposition:smoothing varphi}}
\refnum{Proposition}{proposition:smoothing varphi}. 
Meanwhile define $g_{\varepsilon} \coloneqq g \ast \varphi_{\varepsilon}$,
where $\ast$ is the convolution operator. 
Clearly, $g_{\varepsilon} \in C_{0}^{\infty} (\mathbb{R}, \mathbb{R})$, 
and $g_{\varepsilon} \xrightarrow[\varepsilon \to 0]{pointwise} g$.

So by definition, we have 
$D(g_{\varepsilon}(F_k)) = \frac{d}{dx} g_{\varepsilon}(F_k) DF_k$. 

$\forall \varepsilon > 0$ and $\forall k \in \mathbb{N}$, 
by Minkowski's inequality
we have $\lVert g_{\varepsilon}(F_k) - g_{}(F) \rVert_{L^2(\Omega)}
\leq \lVert g_{\varepsilon}(F_k) - g_{}(F_k) \rVert_{L^2(\Omega)}
+ \lVert g_{}(F_k) - g_{}(F) \rVert_{L^2(\Omega)}$. 

In addition, 
\begin{equation*}
\begin{aligned}
&\lVert \frac{d}{dx}g_{}(F)DF - \frac{d}{dx} g_{\varepsilon}(F_k)DF_k \rVert_{L^2(\Omega, H)} \\
\leq &\lVert \frac{d}{dx}g_{}(F)DF - \frac{d}{dx} g_{}(F_k)DF \rVert_{L^2(\Omega, H)}\\
&+ \lVert \frac{d}{dx} g_{}(F_k)DF - \frac{d}{dx} g_{\varepsilon}(F_k)DF \rVert_{L^2(\Omega, H)} \\
&+ \lVert \frac{d}{dx} g_{\varepsilon}(F_k)DF - \frac{d}{dx} g_{\varepsilon}(F_k)DF_k  \rVert_{L^2(\Omega, H)}
\end{aligned}
\end{equation*}

Because $D$ is well defined, we have $g_{}(F) \in \mathbb{D}^{1, 2}$ and the chain rule. 

\end{proof}

