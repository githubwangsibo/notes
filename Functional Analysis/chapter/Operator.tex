\chapter{Operator}

\section{Basic}

\begin{definition}[kernel]
\end{definition}

\begin{definition}[range/image]
\end{definition}

\begin{definition}[operator norm]
\end{definition}

\begin{theorem}[equivalent definition of operator norm]
The following statements are equivalent: 
\end{theorem}

\begin{proof}
to do
\end{proof}

\begin{warning}
Not all operator has its norm. 
\end{warning}

\section{Linear Operator}

\begin{definition}[linear operator]
\end{definition}

\begin{notation}[set of linear operator]
$\mathcal{L}(X, Y)$
\end{notation}

\begin{definition}[linear bounded operator]
\end{definition}

\begin{notation}[set of linear bounded operator]
$\mathcal{B}(X, Y)$
\end{notation}

\begin{theorem}
Given normed vector spaces $X, Y$ and a linear operator $T \colon X \to Y$. $\ker T$ is a linear subspace in $X$. 
\end{theorem}

\begin{theorem}
Given normed vector spaces $X, Y$ and a linear operator $T \colon X \to Y$. $T(X)$ is a linear subspace in $Y$. 
\end{theorem}

\begin{theorem}
Given normed vector spaces $X, Y$ and a linear operator $T \colon X \to Y$. $T$ is injective $\iff$ $\ker T = \{0\}$. 
\end{theorem}

\begin{theorem}
Given finite normed vector spaces $X, Y$ and a linear operator $T \colon X \to Y$. $T$ is injective $\iff$ $T$ is surjective. 
\end{theorem}

\begin{proof}
to do
\end{proof}

\section{Bounded Operator}
\begin{definition}[bounded operator]
\end{definition}

\begin{theorem}[equivalent definition of linear bounded operator]
Given normed vector spaces $X, Y$ and a linear operator $T \colon X \to Y$. The following statements are equivalent: 
\end{theorem}

\begin{proof}
to do
\end{proof}

\begin{theorem}
Given a normed vector spaces $X, Y$ and a linear bounded operator $T \colon X \to Y$. $\ker T$ is always closed. 
\end{theorem}

\begin{proof}[Sketch of the Proof]
Because every singleton set in a normed vector space is closed. 
\end{proof}

\begin{theorem}
Given normed vector spaces $X, Y, Z$ and operators $T \colon X \to Y, S \colon Y \to Z$. If $S, T$ are bounded, then $\Vert S T \Vert \leq \Vert T \Vert \Vert S \Vert$. 
\end{theorem}

\begin{corollary}
Given normed vector spaces $X, Y$ and an operator $T \colon X \to Y$. $\forall n \in \mathbb{N}$, $\Vert T^n \Vert \leq \Vert T \Vert^n$. 
\end{corollary}

\begin{theorem}
Given any normed vector space $X, Y$ and an operator $T \colon X \to Y$. If $X$ is finite and $T$ is linear, then $T$ is always bounded. 
\end{theorem}

\begin{proof}
to do
\end{proof}

\begin{theorem}[set of bounded linear operators is Banach]
Given normed vector spaces $X, Y$. If $Y$ is Banach, then $\mathcal{B}(X, Y)$ is Banach. 
\end{theorem}

\begin{proof}
We pick any Cauchy sequence $\{T_n\}$ in $\mathcal{B}(X, Y)$. Then, it is obvious that $\forall \varepsilon \in \mathbb{R}^+$, $\exists N(\varepsilon) \in \mathbb{N}$, such that $\Vert T_i - T_j \Vert < \varepsilon$, $\forall i, j > N$.

Also, it is true that $\forall x \in X$, $\forall n, m \in \mathbb{N}$, $\Vert T_n x - T_m x \Vert_{Y} = \Vert (T_n - T_m) x \Vert_{Y} \leq \Vert T_n - T_m \Vert \Vert x \Vert_{X}$. 

So it is true that $\forall x \in X$, $\{T_n x\}$ is a Cauchy sequence in $Y$. Because $Y$ is complete, $\forall x \in X$, $\exists y \in Y$ such that $y = \lim_{n \to +\infty} T_n x$. In another word, there exists a well defined operator $T$: 
\begin{align*}
T \colon X &\to Y \\
x &\mapsto T x = \lim_{n \to +\infty} T_n x
\end{align*}

Define $S \coloneqq \{x \in X \colon \Vert x \Vert_{X} = 1\}$. 

Next, we will prove that $T \in \mathcal{B}(X, Y)$. 

Define $M \coloneqq \sup_{n \in \mathbb{N}} \Vert T_n \Vert$, and it is true that $M < +\infty$. It is known that $\forall \varepsilon \in \mathbb{R}^+$, $\forall x \in X$, $\exists N(\varepsilon, x) \in \mathbb{N}$, such that $\Vert T x - T_n x \Vert_{Y} < \varepsilon$, $\forall n > N$. So $\forall \varepsilon \in \mathbb{R}^+$, $\forall x \in S$, $\forall n > N(\varepsilon, x)$, $\Vert T x \Vert_{Y} = \Vert T x - T_n x + T_n x \Vert_{Y} \leq \Vert T x - T_n x \Vert_{Y} + \Vert T_n x \Vert_{Y} \leq \varepsilon + \Vert T_n \Vert \Vert x \Vert_{X} \leq M + \varepsilon$. So $T$ is bounded. In addition, it is trivial to show that $T$ is linear. 

Next, we will prove that $\lim_{n \to +\infty} \Vert T - T_n \Vert = 0$.

It is true that $\forall n \in \mathbb{N}$, $\Vert T - T_n \Vert_{} = \sup_{x \in S} \Vert (T - T_n)x \Vert_{Y}$. Also, it is true that $\forall x \in S$, $\lim_{n \to +\infty} \Vert (T - T_n)x \Vert_{Y} = 0$. So, $\lim_{n \to +\infty} \Vert T - T_n \Vert = 0$.
\end{proof}

\begin{corollary}
Dual space of any normed vector spaces is a Banach space.
\end{corollary}

\begin{theorem}[extension theorem]
\end{theorem}

\begin{proof}
to do
\end{proof}

\begin{theorem}[open mapping theorem]
\end{theorem}

\begin{proof}
to do
\end{proof}

\begin{theorem}[bounded inverse theorem]
\end{theorem}

\begin{proof}
to do
\end{proof}

\begin{theorem}[uniform boundedness principle]
\end{theorem}

\begin{proof}
to do
\end{proof}

\begin{theorem}[等价范数定理]
\end{theorem}

\begin{proof}
to do
\end{proof}

\begin{definition}[closed graph]
\end{definition}

\begin{theorem}[closed graph theorem]
\end{theorem}

\begin{proof}
to do
\end{proof}

\section{Unbounded Operator}

\section{Compact Operator}

\begin{definition}[compact operator]
\end{definition}

\begin{notation}[set of linear compact operator]
$\mathcal{K}(X, Y)$
\end{notation}

\begin{theorem}
$\mathcal{K}(X, Y)$ is closed in $\mathcal{B}(X, Y)$. 
\end{theorem}

\begin{proof}
to do
\end{proof}

\begin{corollary}
Given a sequence of compact operator $\{K_n\}$, If it is convergent to $K$. then $K$ is a compact operator. 
\end{corollary}

\begin{example}[a linear bounded operator might not be compact]
\end{example}

\begin{theorem}
Given a compact operator $K$ and a linear bounded operator $B$. $K B$ and $B K$ are compact operators. 
\end{theorem}

\begin{proof}
to do
\end{proof}

\section{Functional}

\begin{definition}[linear functional]
\end{definition}

\begin{definition}[dual space, linear bounded functional]
\end{definition}

\begin{notation}[set of linear functional]
$\mathcal{L}(X, \mathbb{R})$ or $\mathcal{L}(X, \mathbb{C})$ or $X^*$
\end{notation}

\begin{notation}[set of linear bounded functional]
$\mathcal{B}(X, \mathbb{R})$ or $\mathcal{B}(X, \mathbb{C})$ or $X^*$
\end{notation}

\begin{definition}[reflexive space]
\end{definition}

\begin{theorem}[Hahn-Banach theorem]\label{theorem:Hahn-Banach theorem}
\end{theorem}

\begin{proof}
to do
\end{proof}

\section{Contraction Mapping}

\begin{theorem}[Picard–Lindelöf theorem]
\end{theorem}

\begin{theorem}[Peano existence theorem]
\end{theorem}

\begin{theorem}[Osgood uniqueness theorem]
\end{theorem}


