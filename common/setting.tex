\documentclass[multixcb, 14pt]{../common/amstext-l}
\usepackage{url}
\usepackage{mathrsfs}
\usepackage{hyperref}
\usepackage{centernot}
\usepackage{mathtools}
\hypersetup{
%    bookmarks=true,         % show bookmarks bar?
%    unicode=false,          % non-Latin characters in Acrobat’s bookmarks
%	pdfborder={0 0 0}, 
%    pdftoolbar=true,        % show Acrobat’s toolbar?
%    pdfmenubar=true,        % show Acrobat’s menu?
%    pdffitwindow=false,     % window fit to page when opened
%    pdfstartview={FitH},    % fits the width of the page to the window
%    pdftitle={My title},    % title
%    pdfauthor={Author},     % author
%    pdfsubject={Subject},   % subject of the document
%    pdfcreator={Creator},   % creator of the document
%    pdfproducer={Producer}, % producer of the document
%    pdfkeywords={keyword1, key2, key3}, % list of keywords
%    pdfnewwindow=true,      % links in new PDF window
    colorlinks=true,       % false: boxed links; true: colored links
    linkcolor=[RGB]{0,128,128},          % color of internal links (change box color with linkbordercolor)
%    citecolor=green,        % color of links to bibliography
%    filecolor=cyan,         % color of file links
%    urlcolor=magenta        % color of external links
}
\usepackage[UTF8, scheme=plain, punct=plain, zihao=false]{ctex}

\DeclareMathAlphabet{\mathzc}{OT1}{pzc}{m}{it}

\theoremstyle{plain}
\newtheorem{theorem}{Theorem}[chapter]
\newtheorem{lemma}[theorem]{Lemma}
\newtheorem{corollary}[theorem]{Corollary}
\newtheorem{proposition}[theorem]{Proposition}

\theoremstyle{definition}
\newtheorem{definition}[theorem]{Definition}
\newtheorem{remark}[theorem]{Remark}
\newtheorem{notation}[theorem]{Notation}
\newtheorem{warning}[theorem]{Warning}
\newtheorem{example}[theorem]{Example}
\newtheorem{xca}{Exercise}[section]

\newcommand{\cs}[1]{\texttt{\char`\\#1}}
\newcommand{\cls}[1]{\texttt{#1}}
\newcommand{\pkg}[1]{\texttt{#1}}
\newcommand{\env}[1]{\texttt{#1}}
\newcommand{\opt}[1]{\texttt{[#1]}}

\def\<#1>{$\langle$\textit{#1}$\rangle$}
\newenvironment{exm}{%
  \par
  \begingroup
    \parindent0pt
    \leftskip2\normalparindent
    \obeylines
}{%
    \par
  \endgroup
%  \noindent\ignorespaces
}

\newenvironment{sketch}{%
  \renewcommand{\proofname}{Sketch of the proof}\proof}{\endproof}
  